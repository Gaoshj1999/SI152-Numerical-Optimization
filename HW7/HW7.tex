\documentclass[10pt]{article}
\usepackage[UTF8]{ctex}

\usepackage[utf8]{inputenc} % allow utf-8 input
\usepackage{url}
\usepackage{bm}
\usepackage{amsmath,amscd}
\usepackage{amssymb,array}
\usepackage{amsfonts,latexsym}
\usepackage{graphicx,subfig,wrapfig}
\usepackage{times}
\usepackage{psfrag,epsfig}
\usepackage{verbatim}
\usepackage{tabularx}
\usepackage[pagebackref=true,breaklinks=true,letterpaper=true,colorlinks,bookmarks=false]{hyperref}
\usepackage{cite}
\usepackage{algorithm}
\usepackage{multirow}
\usepackage{caption}
\usepackage{algorithmic}
\usepackage[amsmath,thmmarks]{ntheorem}
\usepackage{listings}
\usepackage{color}


\newtheorem{thm}{Theorem}
\newtheorem{mydef}{Definition}

\DeclareMathOperator*{\rank}{rank}
\DeclareMathOperator*{\trace}{trace}
\DeclareMathOperator*{\acos}{acos}
\DeclareMathOperator*{\argmax}{argmax}


\renewcommand{\algorithmicrequire}{ \textbf{Input:}}     
\renewcommand{\algorithmicensure}{ \textbf{Output:}}
\renewcommand{\mathbf}{\boldsymbol}
\newcommand{\mb}{\mathbf}
\newcommand{\matlab}[1]{\texttt{#1}}
\newcommand{\setname}[1]{\textsl{#1}}
\newcommand{\Ce}{\mathbb{C}}
\newcommand{\Ee}{\mathbb{E}}
\newcommand{\Ne}{\mathbb{N}}
\newcommand{\Se}{\mathbb{S}}
\newcommand{\norm}[2]{\left\| #1 \right\|_{#2}}

\newenvironment{mfunction}[1]{
	\noindent
	\tabularx{\linewidth}{>{\ttfamily}rX}
	\hline
	\multicolumn{2}{l}{\textbf{Function \matlab{#1}}}\\
	\hline
}{\\\endtabularx}

\newcommand{\parameters}{\multicolumn{2}{l}{\textbf{Parameters}}\\}

\newcommand{\fdescription}[1]{\multicolumn{2}{p{0.96\linewidth}}{
		
		\textbf{Description}
		
		#1}\\\hline}

\newcommand{\retvalues}{\multicolumn{2}{l}{\textbf{Returned values}}\\}
\def\0{\boldsymbol{0}}
\def\b{\boldsymbol{b}}
\def\bmu{\boldsymbol{\mu}}
\def\e{\boldsymbol{e}}
\def\u{\boldsymbol{u}}
\def\x{\boldsymbol{x}}
\def\v{\boldsymbol{v}}
\def\w{\boldsymbol{w}}
\def\N{\boldsymbol{N}}
\def\X{\boldsymbol{X}}
\def\Y{\boldsymbol{Y}}
\def\A{\boldsymbol{A}}
\def\B{\boldsymbol{B}}
\def\y{\boldsymbol{y}}
\def\cX{\mathcal{X}}
\def\transpose{\top} % Vector and Matrix Transpose

%\long\def\answer#1{{\bf ANSWER:} #1}
\long\def\answer#1{}
\newcommand{\myhat}{\widehat}
\long\def\comment#1{}
\newcommand{\eg}{{e.g.,~}}
\newcommand{\ea}{{et al.~}}
\newcommand{\ie}{{i.e.,~}}

\newcommand{\db}{{\boldsymbol{d}}}
\renewcommand{\Re}{{\mathbb{R}}}
\newcommand{\Pe}{{\mathbb{P}}}

\hyphenation{MATLAB}

\usepackage[margin=1in]{geometry}

\begin{document}
	
\title{	Numerical Optimization, 2020 Fall\\Homework 7}
\date{Due on 14:59 NOV 26, 2020\\
	请尽量使用提供的tex模板,若手写作答请标清题号并拍照加入文档.
	}
\maketitle

%%%%%--------------------
\section{收敛速率}
分别构造具有次线性,线性,超线性和二阶收敛速率的序列的例子。{\color{red}[10 pts]}

\section{梯度下降法的收敛性分析}
考虑如下优化问题:
\begin{equation}\label{f.prob}
	\min_{\bm{x}\in \mathbb{R}^n} \quad f(\bm{x}),
\end{equation}
其中目标函数$f$满足一下性质:
\begin{itemize}
	\item 对任意$\bm{x}$, $f(\bm{x})\ge \underline{f}$。
	\item $\nabla f$是Lipschitz连续的,即对于任意的$\bm{x}, \bm{y}$,存在$L>0$使得
	$$
	\|\nabla f(\bm{x}) - \nabla f(\bm{y})\|_2 \le L \|\bm{x}- \bm{y}\|_2.
	$$	
\end{itemize}
若采用梯度下降法求解问题\eqref{f.prob},记所产生的迭代点序列为$\{\bm{x}^k\}$。迭代点的更新为$\bm{x}^{k+1} \leftarrow \bm{x}^k + \alpha^k\bm{d}^k$。试证明以下问题。
\begin{enumerate}
	\item[(i)] 在一点$\bm{x}^k$处给定一个下降方向$\bm{d}^k$,即$\bm{d}^k$满足$\left \langle \nabla f(\bm{x}^k), \bm{d}^k \right \rangle <0$。试证明:对于充分小的$\alpha>0$,有$f(\bm{x}+\alpha\bm{d}^k)<f(\bm{x}^k)$成立。{\color{red}[10 pts]}
	\item[(ii)] 假设存在$\delta>0$使得$-\frac{\left \langle \nabla f(\bm{x}^k), \bm{d}^k \right \rangle}{\|\nabla f(\bm{x}^k)\|_2\|\bm{d}^k\|_2}>\delta$。证明回溯线搜索会有限步终止,并给出对应步长$\alpha^k$的下界。{\color{red}[10 pts]}
	\item[(iii)] 根据上一问结果证明$\lim_{k\to\infty}\|\nabla f(\bm{x}^k)\|_2= \bm{0}$。{\color{red}[10 pts]}
	\item[(iv)] 令$\bm{d}^k=-\nabla f(\bm{x}^k)$,采用固定步长$\alpha^k\equiv \alpha = \frac{1}{L}$。试证明该设定下梯度下降法的全局收敛性。{\color{red}[20 pts]}
\end{enumerate}

\section{编程题}
考虑求解如下优化问题:
\begin{equation}\label{f.rosen}
\min_{x_1, x_2} \quad 100(x_2 - x_1^2)^2 + (1-x_1)^2.
\end{equation}
分别用\textbf{梯度下降法}和\textbf{牛顿法}结合Armijo回溯搜索编程求解该问题。分别考虑用$\bm{x}^0=[1.2, 1.2]^T$和$\bm{x}^0=[-1.2, 1]^T$(较困难)作为初始点启动算法。

\textbf{要求:对于两种初始点,分别画出两种算法步长$\alpha^k$和$\|\nabla f(\bm{x}^k)\|_\infty$随迭代步数$k$变化的曲线。}({\color{red}编程可使用matlab或python完成,请将代码截图贴在该文档中。}) {\color{red}[40pts]}

(Hint: 步长初始值$\alpha_0=1$,参数$c_1$可选为$10^{-4}$,终止条件为$\|\nabla f(\bm{x}^k)\|_\infty\le 10^{-4}$.)



\end{document} 