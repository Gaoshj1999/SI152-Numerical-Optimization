\documentclass[10pt]{article}
\usepackage[UTF8]{ctex}

\usepackage[utf8]{inputenc} % allow utf-8 input

\usepackage{amsmath,amscd}
\usepackage{amssymb,array}
\usepackage{amsfonts,latexsym}
\usepackage{graphicx,subfig,wrapfig}
\usepackage{times}
\usepackage{psfrag,epsfig}
\usepackage{verbatim}
\usepackage{tabularx}
\usepackage[pagebackref=true,breaklinks=true,letterpaper=true,colorlinks,bookmarks=false]{hyperref}
\usepackage{cite}
\usepackage{algorithm}
\usepackage{multirow}
\usepackage{caption}
\usepackage{algorithmic}
\usepackage[amsmath,thmmarks]{ntheorem}
\usepackage{listings}
\usepackage{color}


\newtheorem{thm}{Theorem}
\newtheorem{mydef}{Definition}

\DeclareMathOperator*{\rank}{rank}
\DeclareMathOperator*{\trace}{trace}
\DeclareMathOperator*{\acos}{acos}
\DeclareMathOperator*{\argmax}{argmax}


\renewcommand{\algorithmicrequire}{ \textbf{Input:}}     
\renewcommand{\algorithmicensure}{ \textbf{Output:}}
\renewcommand{\mathbf}{\boldsymbol}
\newcommand{\mb}{\mathbf}
\newcommand{\matlab}[1]{\texttt{#1}}
\newcommand{\setname}[1]{\textsl{#1}}
\newcommand{\Ce}{\mathbb{C}}
\newcommand{\Ee}{\mathbb{E}}
\newcommand{\Ne}{\mathbb{N}}
\newcommand{\Se}{\mathbb{S}}
\newcommand{\norm}[2]{\left\| #1 \right\|_{#2}}

\newenvironment{mfunction}[1]{
	\noindent
	\tabularx{\linewidth}{>{\ttfamily}rX}
	\hline
	\multicolumn{2}{l}{\textbf{Function \matlab{#1}}}\\
	\hline
}{\\\endtabularx}

\newcommand{\parameters}{\multicolumn{2}{l}{\textbf{Parameters}}\\}

\newcommand{\fdescription}[1]{\multicolumn{2}{p{0.96\linewidth}}{
		
		\textbf{Description}
		
		#1}\\\hline}

\newcommand{\retvalues}{\multicolumn{2}{l}{\textbf{Returned values}}\\}
\def\0{\boldsymbol{0}}
\def\b{\boldsymbol{b}}
\def\bmu{\boldsymbol{\mu}}
\def\e{\boldsymbol{e}}
\def\u{\boldsymbol{u}}
\def\x{\boldsymbol{x}}
\def\v{\boldsymbol{v}}
\def\w{\boldsymbol{w}}
\def\N{\boldsymbol{N}}
\def\X{\boldsymbol{X}}
\def\Y{\boldsymbol{Y}}
\def\A{\boldsymbol{A}}
\def\B{\boldsymbol{B}}
\def\y{\boldsymbol{y}}
\def\cX{\mathcal{X}}
\def\transpose{\top} % Vector and Matrix Transpose

%\long\def\answer#1{{\bf ANSWER:} #1}
\long\def\answer#1{}
\newcommand{\myhat}{\widehat}
\long\def\comment#1{}
\newcommand{\eg}{{e.g.,~}}
\newcommand{\ea}{{et al.~}}
\newcommand{\ie}{{i.e.,~}}

\newcommand{\db}{{\boldsymbol{d}}}
\renewcommand{\Re}{{\mathbb{R}}}
\newcommand{\Pe}{{\mathbb{P}}}

\hyphenation{MATLAB}

\usepackage[margin=1in]{geometry}

\begin{document}
	
\title{	Numerical Optimization, 2020 Fall\\Homework 1}
\date{Due on 14:59 Sep 15, 2020\\
	请尽量使用提供的tex模板,画图部分可手绘拍照加入文档.
	}
\maketitle

%%%%%--------------------
\section{优化问题的应用}
给出目前业界线性规划的一个应用场景.介绍模型(变量、约束、目标).一般的规模是多大?


\section{将下述问题建模成线性规划问题} 

一个原油精练场有8百万桶原油A和5百万桶原油B用以安排下个月的生产.可用这些资源来生产售价为38元/桶的汽油,或者生产售价为33元/桶的民用燃料油.有三种生产过程可供选择,各自的生产参数如下:
\begin{figure}[ht]
	\centering
	\includegraphics[width=0.5\linewidth]{prob1.png}
	%\caption{Accuracy demo.}
	\label{fig.prob1}
\end{figure}
除成本外,所有的量均以桶为单位.例如,对于第一个过程而言,利用3桶原油A和5桶原油B可以生产4桶汽油和3桶民用燃料油,成本为51元.表格中的成本指总的成本(即原油成本和生产过程的成本).将此问题建模成线性规划,其能使管理者极大化下个月的净利润.


\section{线性规划的等价转换}
\begin{enumerate}
	\item[(i)]
	考虑如下线性回归问题.令$(x_1, y_1), (x_2, y_2),\cdots,(x_n, y_n)$为样本点和对应标签, a和b为线性模型的参数.线性回归模型可表示为$y_i = a x_i + b,\ i=1,\cdots,n$.
	用$L_{\infty}$范数作为该线性模型的损失函数,则对应的数学规划问题可建模为:
	\begin{equation}\label{prob.Linf}
	\min_{a, b}\ \max_i\ |y_i - (a x_i + b)|.
	\end{equation}
	将\eqref{prob.Linf}改写成等价的线性规划模型.

	\item[(ii)]
	极小化如下绝对值和问题:
	\begin{equation}\label{prob.sumabs}
	\begin{aligned}
		\min_{x_1, x_2}\quad &|x_1| + |x_2|\\
		s.t.\quad            &x_1 + 3x_2 \ge 5\\
							 &2x_1 + x_2 \ge 5.
	\end{aligned}
	\end{equation}
	\begin{enumerate}
		\item[(a)] 引入新变量$x_1^+,x_1^-, x_2^+,x_2^-$,将问题\eqref{prob.sumabs}转换为线性规划问题.
		\item[(b)] 分析为何只有当互补条件(即$x_1^+ x_1^-=0, x_2^+ x_2^-=0$)成立时, 问题取得最优解.
		\item[(c)] 图解法求解问题\eqref{prob.sumabs}.
	\end{enumerate}
\end{enumerate}
	
\end{document} 